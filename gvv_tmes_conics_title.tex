% tMESguide.tex
% v4.2 released May 2008

\documentclass[]{tMES2e}

\usepackage{amsmath}
\usepackage{amssymb}
%\usepackage{amsthm}

\providecommand{\abs}[1]{\lvert#1\rvert}
\providecommand{\sbrak}[1]{\ensuremath{{}\left[#1\right]}}
\providecommand{\lsbrak}[1]{\ensuremath{{}\left[#1\right.}}
\providecommand{\rsbrak}[1]{\ensuremath{{}\left.#1\right]}}
\providecommand{\brak}[1]{\ensuremath{\left(#1\right)}}
\providecommand{\lbrak}[1]{\ensuremath{\left(#1\right.}}
\providecommand{\rbrak}[1]{\ensuremath{\left.#1\right)}}
\providecommand{\cbrak}[1]{\ensuremath{\left\{#1\right\}}}
\providecommand{\lcbrak}[1]{\ensuremath{\left\{#1\right.}}
\providecommand{\rcbrak}[1]{\ensuremath{\left.#1\right\}}}

\newcommand*{\permcomb}[4][0mu]{{{}^{#3}\mkern#1#2_{#4}}}
\newcommand*{\perm}[1][-3mu]{\permcomb[#1]{P}}
\newcommand*{\comb}[1][-1mu]{\permcomb[#1]{C}}
%\DeclareMathOperator{\cosh}{cosh}
%\DeclareMathOperator{\sinh}{sinh}
%\DeclareMathOperator{\csch}{csch}
\begin{document}

\markboth{}{Algebraic proofs for converse theorems for  a cyclic quadrilateral}

%\markboth{G V V Sharma}{Algebraic proofs for converse theorems for  a cyclic quadrilateral}

\articletype{}

\title{Algebraic proofs for converse theorems for  a cyclic quadrilateral}
\author{G V V Sharma$^{\ast}$\thanks{ $^{\ast}$Email: gadepall@ee.iith.ac.in
\vspace{6pt}} \\\vspace{6pt}  {\em{EE Department, IIT Hyderabad,\\ Kandi 502285, India}}}




\maketitle

\begin{abstract}
Proofs of some theorems related to cyclic quadrilaterals are provided using coordinate geometry and trigonometry.  Through this approach, constructions and proofs using contradiction are avoided. 
\begin{keywords}
cyclic quadrilateral, coordinate geometry, trigonometry
\end{keywords}\bigskip


\end{abstract}


%\section{Introduction}
%The hyperbolic functions are defined as \cite{loney}
%\begin{align}
%\sinh x = \frac{e^{x}-e^{-x}}{2}
%\cosh x = \frac{e^{x}+e^{-x}}{2}
%\end{align}
%and 
%\begin{align}
%\sinh^n x = \brak{\frac{e^{x}-e^{-x}}{2}}^n
%\end{align}
%can be expressed as a linear combination of exponential functions involving binomial coefficients \cite{hall}.  This property, along with the series expansion for $e^x$ is used for expressing several new binomial sums in closed form. 
%
%\section{Main Results}
%The following examples describe the process of obtaining the binomial sums.
%%
%\begin{example}
%%
%To show that
%%
%\begin{multline}
%\comb{n}{n}\brak{n+1}^{n+1}-\comb{n}{n-1}n^{n+1} + \comb{n}{n-2}\brak{n-1}^{n+1} + \dots +\brak{-1}^n\comb{n}{0} 1^{n+1} 
%\\
%= \frac{\brak{n+2}!}{2}
%%\frac{1+\sum_{r=1}^{n}r\, \perm{r}{r}}{2}
%\label{eq:comb_1}
%\end{multline}
%%
%\end{example}
%\proof
%Using the binomial theorem, 
%%
%\begin{align}
%\begin{split}
%\brak{e^{x}-1}^n &=\comb{n}{n}e^{nx} - \comb{n}{n-1}e^{\brak{n-1}x} + \comb{n}{n-2}e^{\brak{n-2}x} + \dots + \brak{-1}^n\comb{n}{0}\\
%\Rightarrow e^x\brak{e^{x}-1}^n &=\comb{n}{n}e^{\brak{n+1}x} - \comb{n}{n-1}e^{nx} + \comb{n}{n-2}e^{\brak{n-1}x} + \dots + \brak{-1}^n\comb{n}{0}e^x %\\
%% \brak{1 + x + \frac{x^2}{2!}+\dots} \brak{x+\frac{x^2}{2!}+\frac{x^3}{3!}+\dots}  &=\comb{n}{n}e^{\brak{n+1}x} - \comb{n}{n-1}e^{nx} + \comb{n}{n-2}e^{\brak{n-1}x} + \dots + \brak{-1}^n\comb{n}{0}e^x
%\end{split}
%\label{eq:comb_perm1}
%\end{align}
%Expanding the above as a power series,
%\begin{multline}
% \brak{1 + x + \frac{x^2}{2!}+\dots} \brak{x+\frac{x^2}{2!}+\frac{x^3}{3!}+\dots}^n  
%\\
%=\comb{n}{n}e^{\brak{n+1}x} - \comb{n}{n-1}e^{nx} + \comb{n}{n-2}e^{\brak{n-1}x} + \dots + \brak{-1}^n\comb{n}{0}e^x
%\label{eq:comb_perm2}
%\end{multline}
%%
%Equating the coefficients of $x^{n+1}$  on both sides of \eqref{eq:comb_perm1}
%%
%\begin{multline}
%\comb{n}{n}\frac{\brak{n+1}^{n+1}}{\brak{n+1}!}-\comb{n}{n-1} \frac{n^{n+1}}{\brak{n+1}!}  + \comb{n}{n-2}\frac{\brak{n-1}^{n+1}}{\brak{n+1}!} + \dots +\brak{-1}^n\comb{n}{0} \frac{1^{n+1}}{\brak{n+1}!} = \frac{n+2}{2}
%%\\
%%\implies \comb{n}{n} \brak{n+1}^{n+1} - \comb{n}{n-1} \brak{n}^{n+1} + \comb{n}{n-2} \brak{n+1}^{n-1} + \dots + \brak{-1}^n\comb{n}{0} \brak{1}^{n+1} &= \frac{\brak{n+1}!}{2}
%\end{multline}
%%
%resulting in \eqref{eq:comb_1}.
%%The R.H.S. of \eqref{eq:comb_1} can be expressed as
%%\begin{align}
%%\frac{1 + \sum_{r=1}^{n}r \, \perm{r}{r}}{2} = \frac{1+\sum_{r=1}^{n}\cbrak{\brak{r+1}!-r!}}{2} = \frac{\brak{n+1}!}{2}
%%\end{align}
%%
%%
%\begin{example}
%Show that
%\begin{align}
%\comb{n}{n}n^{n+2}-\comb{n}{n-1}\brak{n-1}^{n+2} + \comb{n}{n-2}\brak{n-2}^{n+2} + \dots  = \frac{n\brak{3n+1}\brak{n+2}!}{3!}
%\label{eq:comb_2}
%\end{align}
%%
%\end{example}
%\proof
%Using the binomial expansion, 
%\begin{align}
%e^{nx}\brak{e^x-e^{-x}}^n = \comb{n}{n}e^{2nx}- \comb{n}{n-1}e^{\brak{2n-2}x} + \comb{n}{n-2}e^{\brak{2n-4}x} - \dots
%\label{eq:comb2_1}
%\end{align}
%and expanding the L.H.S. as a series,
%\begin{align}
%e^{nx}\brak{e^x-e^{-x}}^n &=  \brak{1 + nx + \frac{n^2x^2}{2!}+\dots} \brak{2x+\frac{2x^3}{3!}+\dots}^n  
%\\
%&= 2^n\brak{1 + nx + \frac{n^2x^2}{2!}+\dots} \brak{x^n+n\frac{x^{n+2}}{3!}+\dots} .
%\label{eq:comb2_2}
%\end{align}
%From \eqref{eq:comb2_1} and \eqref{eq:comb2_2},
%\begin{multline}
%2^n\brak{1 + nx + \frac{n^2x^2}{2!}+\dots} \brak{x^n+n\frac{x^{n+2}}{3!}+\dots}  
%\\
%= \comb{n}{n}e^{2nx}- \comb{n}{n-1}e^{\brak{2n-2}x} + \comb{n}{n-2}e^{\brak{2n-4}x} - \dots
%\label{eq:comb2_3}
%\end{multline}
%Collecting coefficients of $x^{n+2}$ on both sides of \eqref{eq:comb2_3},
%\begin{multline}
%\frac{2^n}{\brak{n+2}!}\cbrak{\comb{n}{n}n^{n+2}- \comb{n}{n-1}\brak{n-1}^{\brak{n+2}} + \comb{n}{n-2}\brak{n-1}^{\brak{n+2}x} - \dots} 
%\\
%= 2^n\brak{\frac{n}{3!}+\frac{n^2}{2!}}
%\label{eq:comb2_4}
%\end{multline}
%%
%which, upon simplification, yields \eqref{eq:comb_2}
%
%%
%\begin{example}
%\begin{align}
%\comb{n}{n}\brak{n+1}^{n+2}-\comb{n}{n-1}\brak{n-1}^{n+2} + \comb{n}{n-2}\brak{n-3}^{n+2} + \dots 
%%+\brak{-1}^n\comb{n}{0} 1^{n+1} 
%=2^n\frac{\brak{n+3}!}{3!}
%%=\frac{2^n}{3!} \cbrak{1+\sum_{r=1}^{n+3}r\, \perm{r}{r}}
%\label{eq:comb_3}
%\end{align}
%\end{example}
%%
%%
%\proof Following the approach in \eqref{eq:comb2_1}, we have the identity
%\begin{align}
%e^{x}\brak{e^x-e^{-x}}^n = \comb{n}{n}e^{\brak{n+1}x}- \comb{n}{n-1}e^{\brak{n-1}x} + \comb{n}{n-2}e^{\brak{n-2}x} - \dots.
%\label{eq:comb3_1}
%\end{align}
%Expanding the L.H.S. above as a power series,
%\begin{align}
%e^{x}\brak{e^x-e^{-x}}^n  &= \brak{2x}^n \brak{1 + x + \frac{x^2}{2!}+\dots} \brak{1+\frac{x^2}{3!}+\frac{x^4}{5!}+\dots}^n  
%\\
%&=2^nx^n \brak{1 + x + \frac{x^2}{2!}+\dots} \brak{1+\frac{nx^2}{3!}+\dots}
%\label{eq:comb3_2}
%\end{align}
%From \eqref{eq:comb3_1} and \eqref{eq:comb3_2}, equating the coefficients of $x^n$ on both sides, 
%\begin{align}
%2^n\frac{\brak{n+3}}{3!} &= \comb{n}{n}\frac{\brak{n+1}^{n+2}}{\brak{n+2}!}-\comb{n}{n-1}\frac{\brak{n-1}^{n+2}}{\brak{n+2}!}+\comb{n}{n-2}\frac{\brak{n-3}^{n+2}}{\brak{n+2}!}-\dots
%%\\
%%\implies 
%%2^n\frac{\brak{n+3}!}{3!} &= \comb{n}{n}{\brak{n+1}^{n+2}}-\comb{n}{n-1}{\brak{n-1}^{n+2}}+\comb{n}{n-2}{\brak{n-3}^{n+2}}-\dots
%\label{eq:comb3_3}
%\end{align}
%yielding \eqref{eq:comb_3} after simplification.
%\begin{example}
%\begin{multline}
%\comb{n}{n}\brak{n^2+1}^{n+2}-\comb{n}{n-1}\brak{n^2-2n+11}^{n+2} + \comb{n}{n-2}\brak{n^2-4n+1}^{n+2} - \dots  
%\\
%= \frac{\brak{n^3+3}\brak{2n}^n\brak{n+2}!}{3!}
%\label{eq:comb_4}
%\end{multline}
%\end{example}
%\proof
%We have the  binomial expansion, 
%\begin{align}
%e^{x}\brak{e^{nx}-e^{-nx}}^n = \comb{n}{n}e^{\brak{n^2+1}x}- \comb{n}{n-1}e^{\brak{n^2-2n+1}x} + \comb{n}{n-2}e^{\brak{n^2-4n+1}x} - \dots 
%\label{eq:comb_4_1}
%\end{align}
%Expanding the L.H.S. as an infinite series,
%%
%\begin{align}
%e^{x}\brak{e^{nx}-e^{-nx}}^n = \brak{2n}^n\brak{1+x+\frac{x^2}{2!}+\dots}\brak{x^n+\frac{n^3x^{n+2}}{3!}+\dots}
%\label{eq:comb_4_2}
%\end{align}
%Equating the coefficients of $x^{n+2}$ in \eqref{eq:comb_4_1} and \eqref{eq:comb_4_2}
%\begin{multline}
%\frac{1}{\brak{n+2}!}\lcbrak{\comb{n}{n}\brak{n^2+1}^{n+2}-\comb{n}{n-1}\brak{n^2-2n+11}^{n+2}}
%\\ 
% +\rcbrak{\comb{n}{n-2}\brak{n^2-4n+1}^{n+2} - \dots}  
%= \frac{\brak{n^3+3}\brak{2n}^n}{3!}
%%
%%e^{x}\brak{e^{nx}-e^{-nx}}^n = \comb{n}{n}e^{\brak{n^2+1}x}- \comb{n}{n-1}e^{\brak{n^2-2n+1}x} + \comb{n}{n-2}e^{\brak{n^2-4n+1}x} - \dots 
%\label{eq:comb_4_3}
%\end{multline}
%resulting in \eqref{eq:comb_4}.
%
%\bibliographystyle{tMES}
%\bibliography{chandwani}
%
\end{document}
